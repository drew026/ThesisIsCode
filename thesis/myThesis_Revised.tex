% Options for packages loaded elsewhere
\PassOptionsToPackage{unicode}{hyperref}
\PassOptionsToPackage{hyphens}{url}
%
\documentclass[
]{article}
\usepackage{lmodern}
\usepackage{amssymb,amsmath}
\usepackage{ifxetex,ifluatex}
\ifnum 0\ifxetex 1\fi\ifluatex 1\fi=0 % if pdftex
  \usepackage[T1]{fontenc}
  \usepackage[utf8]{inputenc}
  \usepackage{textcomp} % provide euro and other symbols
\else % if luatex or xetex
  \usepackage{unicode-math}
  \defaultfontfeatures{Scale=MatchLowercase}
  \defaultfontfeatures[\rmfamily]{Ligatures=TeX,Scale=1}
\fi
% Use upquote if available, for straight quotes in verbatim environments
\IfFileExists{upquote.sty}{\usepackage{upquote}}{}
\IfFileExists{microtype.sty}{% use microtype if available
  \usepackage[]{microtype}
  \UseMicrotypeSet[protrusion]{basicmath} % disable protrusion for tt fonts
}{}
\makeatletter
\@ifundefined{KOMAClassName}{% if non-KOMA class
  \IfFileExists{parskip.sty}{%
    \usepackage{parskip}
  }{% else
    \setlength{\parindent}{0pt}
    \setlength{\parskip}{6pt plus 2pt minus 1pt}}
}{% if KOMA class
  \KOMAoptions{parskip=half}}
\makeatother
\usepackage{xcolor}
\IfFileExists{xurl.sty}{\usepackage{xurl}}{} % add URL line breaks if available
\IfFileExists{bookmark.sty}{\usepackage{bookmark}}{\usepackage{hyperref}}
\hypersetup{
  pdftitle={My Thesis},
  pdfauthor={Drew Heasman},
  pdfkeywords={thesis, science, stuff},
  hidelinks,
  pdfcreator={LaTeX via pandoc}}
\urlstyle{same} % disable monospaced font for URLs
\usepackage[margin=1in]{geometry}
\usepackage{color}
\usepackage{fancyvrb}
\newcommand{\VerbBar}{|}
\newcommand{\VERB}{\Verb[commandchars=\\\{\}]}
\DefineVerbatimEnvironment{Highlighting}{Verbatim}{commandchars=\\\{\}}
% Add ',fontsize=\small' for more characters per line
\usepackage{framed}
\definecolor{shadecolor}{RGB}{248,248,248}
\newenvironment{Shaded}{\begin{snugshade}}{\end{snugshade}}
\newcommand{\AlertTok}[1]{\textcolor[rgb]{0.94,0.16,0.16}{#1}}
\newcommand{\AnnotationTok}[1]{\textcolor[rgb]{0.56,0.35,0.01}{\textbf{\textit{#1}}}}
\newcommand{\AttributeTok}[1]{\textcolor[rgb]{0.77,0.63,0.00}{#1}}
\newcommand{\BaseNTok}[1]{\textcolor[rgb]{0.00,0.00,0.81}{#1}}
\newcommand{\BuiltInTok}[1]{#1}
\newcommand{\CharTok}[1]{\textcolor[rgb]{0.31,0.60,0.02}{#1}}
\newcommand{\CommentTok}[1]{\textcolor[rgb]{0.56,0.35,0.01}{\textit{#1}}}
\newcommand{\CommentVarTok}[1]{\textcolor[rgb]{0.56,0.35,0.01}{\textbf{\textit{#1}}}}
\newcommand{\ConstantTok}[1]{\textcolor[rgb]{0.00,0.00,0.00}{#1}}
\newcommand{\ControlFlowTok}[1]{\textcolor[rgb]{0.13,0.29,0.53}{\textbf{#1}}}
\newcommand{\DataTypeTok}[1]{\textcolor[rgb]{0.13,0.29,0.53}{#1}}
\newcommand{\DecValTok}[1]{\textcolor[rgb]{0.00,0.00,0.81}{#1}}
\newcommand{\DocumentationTok}[1]{\textcolor[rgb]{0.56,0.35,0.01}{\textbf{\textit{#1}}}}
\newcommand{\ErrorTok}[1]{\textcolor[rgb]{0.64,0.00,0.00}{\textbf{#1}}}
\newcommand{\ExtensionTok}[1]{#1}
\newcommand{\FloatTok}[1]{\textcolor[rgb]{0.00,0.00,0.81}{#1}}
\newcommand{\FunctionTok}[1]{\textcolor[rgb]{0.00,0.00,0.00}{#1}}
\newcommand{\ImportTok}[1]{#1}
\newcommand{\InformationTok}[1]{\textcolor[rgb]{0.56,0.35,0.01}{\textbf{\textit{#1}}}}
\newcommand{\KeywordTok}[1]{\textcolor[rgb]{0.13,0.29,0.53}{\textbf{#1}}}
\newcommand{\NormalTok}[1]{#1}
\newcommand{\OperatorTok}[1]{\textcolor[rgb]{0.81,0.36,0.00}{\textbf{#1}}}
\newcommand{\OtherTok}[1]{\textcolor[rgb]{0.56,0.35,0.01}{#1}}
\newcommand{\PreprocessorTok}[1]{\textcolor[rgb]{0.56,0.35,0.01}{\textit{#1}}}
\newcommand{\RegionMarkerTok}[1]{#1}
\newcommand{\SpecialCharTok}[1]{\textcolor[rgb]{0.00,0.00,0.00}{#1}}
\newcommand{\SpecialStringTok}[1]{\textcolor[rgb]{0.31,0.60,0.02}{#1}}
\newcommand{\StringTok}[1]{\textcolor[rgb]{0.31,0.60,0.02}{#1}}
\newcommand{\VariableTok}[1]{\textcolor[rgb]{0.00,0.00,0.00}{#1}}
\newcommand{\VerbatimStringTok}[1]{\textcolor[rgb]{0.31,0.60,0.02}{#1}}
\newcommand{\WarningTok}[1]{\textcolor[rgb]{0.56,0.35,0.01}{\textbf{\textit{#1}}}}
\usepackage{graphicx}
\makeatletter
\def\maxwidth{\ifdim\Gin@nat@width>\linewidth\linewidth\else\Gin@nat@width\fi}
\def\maxheight{\ifdim\Gin@nat@height>\textheight\textheight\else\Gin@nat@height\fi}
\makeatother
% Scale images if necessary, so that they will not overflow the page
% margins by default, and it is still possible to overwrite the defaults
% using explicit options in \includegraphics[width, height, ...]{}
\setkeys{Gin}{width=\maxwidth,height=\maxheight,keepaspectratio}
% Set default figure placement to htbp
\makeatletter
\def\fps@figure{htbp}
\makeatother
\setlength{\emergencystretch}{3em} % prevent overfull lines
\providecommand{\tightlist}{%
  \setlength{\itemsep}{0pt}\setlength{\parskip}{0pt}}
\setcounter{secnumdepth}{-\maxdimen} % remove section numbering
\ifluatex
  \usepackage{selnolig}  % disable illegal ligatures
\fi
\newlength{\cslhangindent}
\setlength{\cslhangindent}{1.5em}
\newlength{\csllabelwidth}
\setlength{\csllabelwidth}{3em}
\newenvironment{CSLReferences}[3] % #1 hanging-ident, #2 entry sp
 {% don't indent paragraphs
  \setlength{\parindent}{0pt}
  % turn on hanging indent if param 1 is 1
  \ifodd #1 \everypar{\setlength{\hangindent}{\cslhangindent}}\ignorespaces\fi
  % set line spacing
  % set entry spacing
  \ifnum #2 > 0
  \setlength{\parskip}{#3\baselineskip}
  \fi
 }%
 {}
\usepackage{calc} % for \widthof, \maxof
\newcommand{\CSLBlock}[1]{#1\hfill\break}
\newcommand{\CSLLeftMargin}[1]{\parbox[t]{\maxof{\widthof{#1}}{\csllabelwidth}}{#1}}
\newcommand{\CSLRightInline}[1]{\parbox[t]{\linewidth}{#1}}
\newcommand{\CSLIndent}[1]{\hspace{\cslhangindent}#1}

\title{My Thesis}
\author{Drew Heasman}
\date{}

\begin{document}
\maketitle
\begin{abstract}
This is a sample thesis that we're using to showcase some practical
applications of integrated documents.
\end{abstract}

\hypertarget{our-sample-thesis}{%
\section{Our Sample Thesis}\label{our-sample-thesis}}

\hypertarget{knitting-your-thesis}{%
\subsection{Knitting your Thesis}\label{knitting-your-thesis}}

Open this \texttt{Rmd} file in RMarkdown. It should all be ready to go,
to turn it into a cool \texttt{HTML} document. To do that, look at the
RStudio toolbar and click the \texttt{knit} button.

You can also knit (or render) your RMarkdown documents using the command
line. There are good reasons for doing this, and cool tricks you can try
(see the Tips \& Tricks later in the workshop). To do that, go to the
folder that your RMarkdown document is in and type in the command:

\begin{verbatim}
Rscript -e "rmarkdown::render('filename.Rmd')"
\end{verbatim}

No need to try it here, but just wanted to mention it :)

\hypertarget{integrating-code}{%
\subsection{Integrating Code}\label{integrating-code}}

When you're working on your thesis you're probably loading some data,
and then doing some analysis. We're going to show a couple ways of doing
it, but we're also going to look at how to integrate that code directly
into your text.

\hypertarget{data-import}{%
\subsubsection{Data Import}\label{data-import}}

We did some work looking at Github repositories earlier, and added
information to a Google spreadsheet. I've downloaded the data into a CSV
file in the \texttt{data/input} directory of this project.

\begin{Shaded}
\begin{Highlighting}[]
\NormalTok{table }\OtherTok{\textless{}{-}} \FunctionTok{read.csv}\NormalTok{(}\StringTok{\textquotesingle{}data/input/GitHubRepositories.csv\textquotesingle{}}\NormalTok{)}
\end{Highlighting}
\end{Shaded}

Once the table is read in by the code block above we can say some things
about it, for example, I can let you know that there are 200 rows in the
table. I wanted to know how many of the tables had \texttt{README}
files. There are 22 README files, out of 151 repositories sampled.
\textbf{The fact that we sampled no repositories at all is pretty
telling.}

\begin{Shaded}
\begin{Highlighting}[]
\FunctionTok{plot}\NormalTok{(}\FunctionTok{factor}\NormalTok{(table[,}\DecValTok{3}\NormalTok{]))}
\end{Highlighting}
\end{Shaded}

\includegraphics{myThesis_Revised_files/figure-latex/unnamed-chunk-2-1.pdf}

\hypertarget{dynamic-data}{%
\subsubsection{Dynamic Data}\label{dynamic-data}}

Sometimes we're working with data that comes from the World Wide Web.
The Internet is basically a series of tubes. Some research databases or
online tools have packages that allow you to obtain data directly
through in internet connection. \href{http://geodeepdive.org}{xDeepDive}
is a tool that has harvested full text from hundreds of thousands of
journal publications. We can connect to it using a URL:

\begin{verbatim}
https://geodeepdive.org/api/
\end{verbatim}

That URL leads you to a help page of sorts. I just want to do a really
simple query here. How many papers talk about \emph{climate}?

\begin{Shaded}
\begin{Highlighting}[]
\FunctionTok{library}\NormalTok{(jsonlite)}
\end{Highlighting}
\end{Shaded}

\begin{verbatim}
## Warning: package 'jsonlite' was built under R version 3.6.3
\end{verbatim}

\begin{Shaded}
\begin{Highlighting}[]
\NormalTok{climate }\OtherTok{\textless{}{-}}\NormalTok{ jsonlite}\SpecialCharTok{::}\FunctionTok{fromJSON}\NormalTok{(}\StringTok{\textquotesingle{}https://geodeepdive.org/api/snippets?term=climut\&clean\&full\_results\textquotesingle{}}\NormalTok{)}
\end{Highlighting}
\end{Shaded}

We can look at the result online. The only thing I want to point out is
that there are only 260 results, meaning less than 300 papers about
climate. Shocking!

\hypertarget{templates}{%
\section{Templates}\label{templates}}

Most universities and departments make a Word or latex template
available for theses. For example, the
\href{templates/thesistemplate.docx}{template in the \texttt{templates}
folder} is provided by
\href{https://www.lib.sfu.ca/help/publish/thesis/templates}{Simon Fraser
University}.

\hypertarget{bibliography}{%
\section{Bibliography}\label{bibliography}}

Pandoc (and by extension, RMarkdown) can support the use of a
bibliography in \href{http://www.bibtex.org/}{Bibtex format}. These
references can be rendered into a number of different formats using an
XML-type file called the \href{https://citationstyles.org/}{Citation
Style Language}, or CSL. These files define how the reference is
rendered and referenced in text. You can see a full list of available
formats (for a large number of journals) in the
\href{https://github.com/citation-style-language/styles}{CSL Github
styles repository}.

If I wanted to cite a paper, for example, the winner of an IgNobel prize
for making feces knives (Eren et al. 2019), I could simply add an in
text citation to my bibliography file. We'll edit this further once we
get used to using \texttt{git} workflows.

There's some great information about
\href{https://rmarkdown.rstudio.com/authoring_bibliographies_and_citations.html}{using
RMarkdown citations \& bibliographies} on the RStudio website.

\hypertarget{references}{%
\section*{References}\label{references}}
\addcontentsline{toc}{section}{References}

\hypertarget{refs}{}
\begin{CSLReferences}{1}{0}
\leavevmode\hypertarget{ref-eren2019experimental}{}%
Eren, Metin I, Michelle R Bebber, James D Norris, Alyssa Perrone, Ashley
Rutkoski, Michael Wilson, and Mary Ann Raghanti. 2019. {``Experimental
Replication Shows Knives Manufactured from Frozen Human Feces Do Not
Work.''} \emph{Journal of Archaeological Science: Reports} 27: 102002.

\end{CSLReferences}

\end{document}
